The big data team focuses on analyzing data in order to support several business decisions.

In my case, the big data team was involved in providing support to the portfolio management activities (described in section \ref{section:users:portfolio_management}).
These activities help both the bidding processes and the portfolio optimization decisions.

The former activity has the goal of helping placing more accurate bids, while the latter has the goal of choosing plants which have a more stable production, which in turn means having a more accurate prediction.
    
\subsection{Needs}
    The big data team needs to use the data warehouse to predict the production of renewable energy plants (i.e. photo-voltaic and wind plants).
    
    Each power plant has to declare how much energy it will produce in advance.
    If the amount declared is less than the energy actually produced, the company will face a fine.
    
    The main issue arises from the fact that renewable sources don't provide the same amount of energy every day, since their production depends on many factors.
    For example, a wind farm doesn't produce any energy on a calm day and produces a lot of energy in a short amount of time if there is a particularly strong wind.
    
\subsection{Current situation}
    \reword At the moment the business department is paying a lot of money in fines, since their predicted energy generation amount is much less than the actual production.
    
    The cause of this errors isn't the mathematical model used but the low quality of the data used to predict the outcome.
    
    The main issues related to data quality are:
    \paragraph{Inconsistent plant names}
        There is no quality assurance control when receiving plant names from providers.
        These names are most likely input manually by clients, since we noticed they sometimes contain typos.
        
        For example a specific UP\footnote{\textit{Unità di produzione} (Production Unit). An UP is a plant which produces electric energy.}, called \textit{UP\_ParcoEolico\_1} was one time named \textit{UP\_ParcoElico\_1}.
        
        Table \ref{tab:sbil:up_names} shows some naming inconsistencies.
        
        \begin{table}
            \centering
            \begin{tabular}{|l|l|}
                \toprule
                Actual name         & Correct name      \\
                \midrule
                IM\_020****         & IM\_20****        \\
                IM\_0S1****         & IM\_S1****        \\
                \midrule
                IM\_036****         & PVI\_036****\_001 \\
                IM\_036****\_01     & PVI\_036****\_001 \\
                \midrule
                UP\_PARCOELICO\_1    & UP\_PARCOEOLICO\_1 \\
                UP\_PARCOEOLICO\_1   & UP\_PARCOEOLICO\_1 \\
                \bottomrule
            \end{tabular}
            \caption{Some UP naming inconsistencies}
            \label{tab:sbil:up_names}
        \end{table}

    \paragraph{Different date conventions}
        Each provider handles DST changes in a different way\footnote{For more information on DST changes and its related problems, see section \ref{etl:problems:dst}.}.
        The main issues are that there are no normalization actions in-place and that all the different providers are stored in the same table.
        
        This causes a problem when extracting data from the table, since each provider needs \textit{ad hoc} logic in the query, which not only makes it harder to read, but also negatively impacts performance.
        
        As an example, a query needed to extract information from table containing data from two weather providers, which we will call \textit{Provider A} and \textit{Provider B}\footnote{The actual names of the providers are trade secrets and, as a consequence, cannot appear in this document.}.
        However the two providers handled DST in different ways.
        Upon further analysis, we discovered that even a single provider doesn't have a consistent approach across all years.
        
        In detail, we have:
            \begin{itemize}
                \item Lack of consistency across providers
                \item Lack of consistency across all years of the same provider
            \end{itemize}

        For example, the day when DST\footnote{For more information about DST, see section \ref{etl:problems:dst}.} ends we are supposed to have 25 hours, since the hour between 2 and 3 a.m. is repeated twice.
        
        The data provided by \textit{Provider A} are ranged 1 to 25, which is what mathematical models expect, while \textit{Provider B} only provides 24 hours and discards the 25th.
        In this case it is necessary to remap all the hours after 2 a.m., which is when the change occurred.
        
        The behavior of \textit{Provider B} isn't however consistent across all years, since we noticed that the data prior to 2017 have 25 hours.
        As such it is important to take into account the year when deciding whether to remap the hours or not.
        
    \paragraph{Data scattering}
        Currently the queries require data from several different sources.
        
        For example, a consistent number of information are extracted from \textit{Arkive1}, a custom tool which acts as a database, but has many more functionalities.
        However, the data contained in \textit{Arkive1} come from a different  tool, \textit{Trimp}, which is a metering tool\footnote{A metering tool regularly receives values from energy meters and performs some basic operations on them depending on the meter type.
        For example old meters send a cumulative count of how much energy has been consumed since the meter was installed, while newer models send only the amount consumed since the last reading.} developed by a third-party organization.\reword
        
        This high interaction between tools adds an additional layer of complexity to the system, negatively impacting both performance and ease of understanding.\reword
        
    \paragraph{Long execution times}
        A consequence of operations required in queries to normalize the data, such as CASE WHEN, is that the queries take a long time to execute.
        
        For example, this happened to be a bottleneck when exporting a table from one db to another, as the transfer speed proved to be excessively low (it took more than half a day to copy less than half of a table).\reword

\subsection{Expected situation}
    \reword The goal of the data warehouse is to provide a higher level of data quality assurance, thus reducing the error on the predictions and improving query times, since it won't be necessary to perform cleaning operations inside the queries.
