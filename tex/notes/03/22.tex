\section{22/03}
    Riunione con Marini per parlare di DWH retail
    
    I dati sono presi da:
    \begin{itemize}
        \item CRM (Salesforce)
        \item Billing (SAP IS-U)
        \item FI-CA (SAP R/3)
        \item Metering (IFS)
    \end{itemize}
    
    Avviene succesivamente una fase di ETL.
    
    La DWH retail ha 5 livelli.
    \begin{enumerate}
        \item Data loading
        \item
        \item Aggregazione temporale dati
        \item Proiezione nel futuro
            \begin{itemize}
                \item Non e` previsione
                \item Si assume non ci siano cambiamenti nel tempo
                \item Utenti non ci lasciano e non ne arrivano di nuovi
                \item \textit{Milazzum} (dal tizio che l'ha sviluppata)
            \end{itemize}
        \item Serie nel tempo (quindi cubi)
            \begin{itemize}
                \item Usato per uniformare il linguaggio
                \item Report mensile, settimanale parte commerciale
                \item Tipo quanti utenti ci sono (in modo da fornire una descrizione)
            \end{itemize}
    \end{enumerate}
    
    La DWH e` aggiornata giornalmente alle 19, impiega un paio d’ore

    Come frontend si usa clickview (in futuro si passera` a clicksense o powerbi)
	I dati sono presi a da L2, L3, L4
	Si usa come input anche il metering (per analisi orarie, non permesse da DWH)

    Tutti i servizi andranno in cloud.

    \subsection{Uscita da DWH}
        Si usano query manuali e \textit{RADAR}.
        \begin{itemize}
            \item Esecutore query normalizzate
            \item Formato output: (num (0,1,2), txt)
            \item Tramite metadati e` possibile raggruppare query
            \item Identifica anomalie dati, per ritrovarli piu` facilmente.
        \end{itemize}

        Report per segmento cliente, numeri per cliente
	    
	    Mancano pero` alcune informazioni su un paio di clienti al giorno (non si sa chi siano).
    
    \subsection{Portafoglio}
        Misure analizzate:
        \begin{itemize}
            \item chi
            \item dove
            \item cosa, a che prezzo
            \item per quanto tempo
        \end{itemize}
        
        Colonne di interesse:
        \begin{itemize}
            \item delivery
            \begin{itemize}
                \item fatturazione certa
            \end{itemize}
            
            \item delivery adjusted
            \begin{itemize}
                \item fatt certa, entrati * p (.85/.9)
            \end{itemize}
            
            \item contractualized
            \begin{itemize}
                \item tutto cio` che entra
            \end{itemize}
        \end{itemize}

    \subsection{Metriche}
        \begin{itemize}
            \item numeriche
            \item volumi
            \item anzianita`
        \end{itemize}
