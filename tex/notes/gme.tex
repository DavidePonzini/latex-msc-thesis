The Italian Power Exchange (IPEX) is the exchange for electricity and natural gas spot trading in Italy.
It is managed by \textit{Gestore del Mercato Elettrico} or GME.\cite{bib:gme:mpe}

IPEX comprises the following spot markets:
    \begin{itemize}
        \item Day-ahead Market
        \item Intra-day Market
        \item Ancillary Services Market
    \end{itemize}
    
\paragraph{Day-ahead Market}
    The Day-ahead Market, or \textit{Mercato Giorno Prima} (\textbf{MGP}), hosts most of the electricity sale and purchase transactions.
    
    Energy blocks, at hourly granularity, are traded for the next day.
    The trading session starts 9 days before the day of delivery and closes at 12 p.m. the day before the day of delivery, hence the name "day-ahead market".
    
    Bids are valued at the National Single Price, or \textit{Prezzo Unico Nazionale} (PUN), which is the average of the prices of geographical zones, weighted for the quantities purchased in these zones\footnote{PUN: $\sum_{i=1}^{n\_zones} \frac{consumption_i * price_i}{consumption_i}$}.
    
    In case some constraints are violated, zonal prices are used instead\footnote{Most constraints are physical limitations, e.g. a cable line cannot physically transfer more than a certain amount of electricity. If it is necessary to transfer more than this amount, it is necessary to use more than one line, which is usually located in a different zone (hence the need to use zonal prices).}.

\paragraph{Intra-day Market}
    The Intra-Day Market, or \textit{Mercato Infragiornaliero} (MI), allows Market Participants to modify the schedules defined in the MGP by submitting additional supply offers or demand bids.
    
    Sales are performed similarly to MGP, while purchases are executed exclusively at zonal prices.
    
    The MI is comprised of seven sessions, named MI1 to MI7.

\paragraph{Ancillary Services Market}
    The Ancillary Services Market, or \textit{Mercato Servizi di Dispacciamento} (MSD), is the venue where Terna S.p.A. (described in detail in section \ref{section:providers:terna}) procures the resources it requires for managing and monitoring RTN\footnote{National Transmission Network, or \textit{Rete Trasmissione Nazionale}. It covers almost all the energy transmission network in Italy. Terna manages around 98\% of RTN.}.
    
    Terna acts as a central counterparty and accepted offers are remunerated at the price offered (\textit{pay-as-bid}).

    The MSD is comprised of two submarkets, called ex-ante MSD and Balancing Market, or \textit{Mercato Bilanciamento} (MB).
    Both markets take place in multiple sessions.
