% one cloud provider or many?
% switch to prod: all at once or a single component at a time?
% https://blog.newrelic.com/engineering/cloud-migration-checklist/

Before beginning a Cloud migration, it is important to take decisions with respect to several factors, which are going to influence the whole process.
The most relevant are described in what follows.

\subsection{Cloud Platform Choice}
    The first choice is whether to use a single Cloud provider or to split the application across multiple providers.
    Both choices have their own advantages and disadvantages \cite{bib:related_work:migration:10_steps}.
    
    \subsubsection{Single provider}
        Using a single Cloud provider is relatively simple.
        The development team needs to learn just one set of cloud APIs, and the application can take advantage of everything the chosen Cloud provider offers.

        The downside to this approach is vendor lock in.
        Moving the application to a different provider could require just as much effort as the original cloud migration.
        Additionally, having a single cloud provider might negatively impact the company ability to negotiate important terms, such as pricing and SLAs\footnote{
            \textit{Service Level Agreement}.
             It is the minimum level of service that a carrier will deliver to the company per agreement.
             When the service dips below that level, the company can open a repair ticket.
        }, with the cloud provider.
    
    \subsubsection{Multiple providers}
        Using multiple providers can provide several advantages, depending on the chosen model.
        
        \paragraph{Different applications on different Clouds}
            The simplest approach is to run a set of applications in one Cloud provider and a different set in another.
            This approach gives the company increased business leverage with multiple providers, as well as flexibility for where to put applications in the future.
            Each application can also be optimized for the provider on which it runs.
            
        \paragraph{Same application on multiple Clouds}
            A different approach consists of running some parts of a single application on one provider, and other parts on a different one.
            
            In this way, it is possible to effectively use the strengths of each provider.
            Let's assume, for example, that \textit{Provider A} has better AI capabilities than the other providers, while \textit{Provider B} offers cheaper and more efficient storage.
            A solution using this approach would store all data on \textit{Provider B}, and transfer only the information needed for AI models to \textit{Provider A}.
            
            There is, however, a risk tied to this approach.
            The performance of the whole application depends on the performance of each provider, meaning that the slowest provider could become a bottleneck for the whole application.
            
            Similarly, downtime of a single provider could result in the whole application not working as intended.
            
        \paragraph{Cloud-agnostic applications}
            A third possibility is building an application which can run on \textit{any} provider.
            
            There are multiple advantages, such as a higher flexibility when negotiating with providers, as well as the ability to run the same application on multiple providers at the same time.
            The latter allows more dynamic workload balancing, since it is possible to easily shift work between different providers.
            
            The downside is the inability to use each of the strengths of each provider, reducing the benefits of hosting the application in the Cloud.
            
\subsection{Migration Techniques}
    After having decided which and how many providers use, it is necessary to choose how to migrate the application to the Cloud.
    
    There are three main approaches, depending on the integration level of the application with the Cloud \cite{bib:related_work:migration:moving_to_cloud, bib:related_work:migration:10_steps}.
    
    \paragraph{Lift-and-shift}
        The simplest and fastest approach is moving the application \textit{as-is} to the cloud.
        
        With this option, however, the application cannot benefit from each of the Cloud unique services, meaning that the improvements will be reduced.
        
    \paragraph{Lift-and-refit}
        An improved version of the \textit{lift-and-shift} approach.
    
        After the application has been moved to the Cloud, some modifications are made where possible, enabling the application to leverage the strengths of the Cloud provider.
        
        The downsides are that this approach requires more time than \textit{lift-and-shift}, as well as in-depth knowledge of the application.
        Moreover, there is also the risk of increasing the complexity of the application.
        
    \paragraph{Cloud native}
        The best solution, although also the most time expensive, is to rebuild the application from scratch on the Cloud.
        
        In this way it is possible to use each of the Cloud features to the fullest, maximising both performance and readability.
        
        The downside is, however, the large amount of time needed to rewrite the whole application.
        
\subsection{Deployment}
    The last important decision is how and when to switch over the production system from the legacy on-premise solution to the new cloud version.
    
    There are two possible solutions.
    
    \paragraph{All at once}
        The switch is performed once the whole application has been successfully ported to the Cloud.
        
        All services are switched at the same moment from on-premise to Cloud.
        
    \paragraph{Incremental}
        After each part of the application has been migrated, some users move their workload to the Cloud.
        This process is then repeated until all users have successfully moved to the Cloud.
        
        For very large applications, this is the best solution, since it isn't necessary to wait for the whole migration to be complete (which could require years) before using it.