Countries can be referred to in different ways.
As such it is necessary to normalize these information.

\paragraph{Problems}
    Providers refer to countries in two different granularities.
    
    The most generic information indicates the country itself, while the most detailed specifies a smaller zone, which belongs to a specific country.
    These zones are specific to the energy sector and divide a country based both on geographical and energetic factors.
    
    Not all countries are divided in zones.

    \subpar{Countries}
        Most providers use the same notation for expressing countries.
        
        However there are a few exceptions.
        For example, some providers spell countries in a different way or using a different case.
        Some other providers refer to countries using a two-character ISO notation\footnote{
            Standard notation commonly used worldwide.
            For example, Italy is abbreviated `IT' and United States `US'.\\
            For a comprehensive list of country codes, see \cite{bib:country_codes}.
        }.
        
        As such, the data needs to be remapped into a common notation, to preserve consistency amongst all data.
        This remapping is done through an apposite table, such as the one shown in table \ref{tab:dwh:remapping:country}.
        
        \begin{table}
            \centering
            \begin{tabular}{|c|c|}
                \toprule
                Original            & Remapped      \\
                \midrule
                Belgian	            & Belgium       \\
                Belgium	            & Belgium       \\
                \midrule
                Germany             & Germany       \\
                germany\_luxembourg	& Germany       \\
                Germany/Luxembourg	& Germany       \\
                \midrule
                netherlands	        & Netherland    \\
                Netherland	        & Netherland    \\
                \midrule
                PT	                & Portugal      \\
                Portugal	        & Portugal      \\
                \bottomrule
            \end{tabular}
            \caption{Country remappings.}
            \label{tab:dwh:remapping:country}
        \end{table}
        
    \subpar{Country zones}
        Country zones are either referred to directly or by using a particular notation known as \texttt{EIC}\footnote{
            \textit{Energy Identification Codes}.
            These coding scheme is used across Europe to facilitate cross-border exchanges and to efficiently and reliably identify different objects and parties relating to the Internal Energy Market and its operations \cite{bib:enstoe:eic}.
        }.
        
        In the first case no normalization operations are needed, while in the second, it is necessary to remap these codes to both a normalized version of the zone name as well as to the country it belongs to.
        
        \begin{table}
            \centering
            \begin{tabular}{|c|l c|}
                \toprule
                EIC                 & Zone                  & Country   \\
                \midrule
                10YIT-GRTN-----B	& Italy	                & Italy     \\
                10Y1001A1001A73I	& Italy\_NORD           & Italy     \\
                % 10Y1001A1001A70O	& Italy\_CNOR           & Italy     \\
                % 10Y1001A1001A71M	& Italy\_CSUD           & Italy     \\
                10Y1001A1001A788	& Italy\_SUD            & Italy     \\
                10Y1001A1001A75E	& Italy\_SICI	        & Italy     \\
                10Y1001A1001A83F	& Germany               & Germany   \\
                10YDE-ENBW-----N	& Germany\_TransnetBW   & Germany   \\
                \bottomrule
            \end{tabular}
            \caption{EIC codes and remapped values.}
            \label{tab:dwh:remapping:country_zone}
        \end{table}
        
        Table \ref{tab:dwh:remapping:country_zone} shows an example of \texttt{EIC} codes, as well as their remapped values.
        
\paragraph{Solution}
    Two additional tables have been created, containing remapping information for both countries and zones.
    
    These tables are similar to tables \ref{tab:dwh:remapping:country} and \ref{tab:dwh:remapping:country_zone}.
    
    Since not all queries require this kind of data, the remapping are performed only at logical level.
    As such, new queries require to perform join operation with the remap tables.
    
    On the other hand, several views created for specific departments already perform these join operations, simplifying user interaction.
    
    The size of these tables are small (less than 100 rows), so the performance impact is negligible.