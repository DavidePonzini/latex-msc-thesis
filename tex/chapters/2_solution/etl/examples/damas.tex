Damas\footnote{
    Damas is a private web portal managed by Terna S.p.A.
    It provides data related to energy transfers between Italy and bordering countries.
    Information from this website are private, meaning that each company can only see their own energy transfers.
} is an example of a website which requires complex navigation logic.

The website offers a simple interface for selecting a time range and showing the data for that given range.
However, this system presents two limitations.

\paragraph{Limited date range}
    First of all, the maximum date range allowed is 31 days.
    If a user tries to perform a query on a bigger date range, the system shows an error message.
    
    This means that it is necessary to perform 12 queries to download a whole year.

\paragraph{No DST}
    The other problem is that if a date range contains a DST change, the website shows the following error message:

    \begin{center}
        \texttt{
            One of the days in the interval BUSDAY(30.03.2019) and\\ BUSDAYTILL(31.03.2019) contains clock change days.
        }
    \end{center}
    
    As shown by the message, it is impossible to download even a small amount of days, if a DST change occurs between them.

    The solution would be to download these dates individually, while downloading the rest of the data in ranges smaller than 31 days.
    However, since each year has different dates for DST changes, it would be necessary to set up these date manually for each year in the downloader.
    
    In the end, the best solution, even though it's computationally less effective, is to download each day individually.
    In this way there is no need to group dates into ranges and to pay attention to DST changes.
    The download process, on the other hand, is slower than downloading data in a batch.