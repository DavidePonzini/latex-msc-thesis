\paragraph{Programming languages}
    Databricks supports multiple programming languages, even in the same file.
    
    Reply has however decided to write almost exclusively in Python, in order to provide a high level of consistency and readability between all scripts.
    
    Different languages are used only when it is impossible, or very complex, to write some functionalities in Python.

\paragraph{File and Folders}
    Reply created several scripts in Databricks, which are used to download and process all required data.

    Scripts are divided into subfolders, each folder representing a different provider.
    
    \subpar{Common functionalities}
        A folder called "Includes" contains several files, which define methods used by all providers.
    
        These methods, for example:
            \begin{itemize}
                \item Set-up connections to DataLake, which is used to store all data downloaded.
                \item Allow file listing and recovery from DataLake.
                \item Allow recovery of FTP files under certain commonly used criteria.
                \item Allow querying databases.
            \end{itemize}
            
        Since these operations need to be performed for each provider, it makes perfectly sense to store them all in one place.
    
    \subpar{Providers}
        The other folders represent all providers from which data needs to be downloaded.
        
        Each folder contains several scripts.
        Each script downloads a single file and applies some custom logic before storing it on DataLake.
        
    \subpar{LocalUtils}
        Each folder associated with a provider contains a sub-folder called "LocalUtils", which defines some methods used by all scripts related to that provider.
        
        This folder, for example, contains all configuration required to download data from the provider, such as FTP credentials.
        
        These methods are used for all data downloaded by a given provider, explaining the need to put them into a common folder.
        
        However, since each provider is different and requires different download approaches, it makes sense not to expose these scripts to other providers.

    \subpar{Deployment}
        All these scripts are stored under a single folder, named \texttt{ETL}.
        
        Having all the data in one place allows for a quick deployment, since it is simply necessary to download the whole folder and overwrite the old one.
        
        