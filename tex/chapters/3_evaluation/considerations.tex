Moving from a complex structure of on-premise databases and tools is not an easy task.
Many problems are encountered in each phase of the development, starting from understanding the various data sources needed to developing an appropriate structure.

First of all, finding a suitable \textbf{workflow} is not an easy matter.
As it has been described, the original workflow presented by Reply had several issues and limitations.
A part of these problems have been solved by making some changes to the process, but a few still remain.
For example, the history retrieval process is now tested more in depth before deployment of a Sprint.

The \textbf{ETL process} also encountered many difficulties, caused by different factors.
Each provider, for example, exposes data in different formats and requires specific data normalization operations.

Testing the \textbf{quality of the data} is also a complex matter, since energy market data present domain-specific difficulties.
For example, it is impossible to automatically assess the correctness of forecast values, since they are expected to change each time.\newline


Even considering all these problems, we can see a great improvement in both \textbf{simplicity} and \textbf{performance}.

The architecture has become more simple, since now all the data is stored on a centralized Data Warehouse, instead of multiple databases, each with its own insertion logic.
As a consequence, most of the integration software previously used are no longer required.

Performance is, however, the most important aspect, since it was the main cause for the migration.
From the test results, we can notice a consistent reduction in query execution times, as long as the appropriate table structures are chosen.

We can conclude by stating that migrating from an on-premise database to a Cloud-based architecture is a good choice, especially in terms of expected performance, although the process is very complex.